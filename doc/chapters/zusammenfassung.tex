\chapter{Zusammenfassung}
Im Zuge dieser Arbeit ist ein Schlussfolgerungssystem für Hornklauselmengen entstanden. Einfache logische Schlüsse werden durch das System bereits gezogen, zum vollständigen Ersatz von Prolog in der Lehrveranstaltung Angewandte Logik ist es jedoch noch nicht geeignet.

Das Auswerten von arithmetischen Ausdrücken durch vordefinierte Prädikate oder das Vergleichen von Zahlen ist noch nicht möglich. Viele der Übungsaufgaben der Veranstaltung erfordern allerdings diese Möglichkeiten.

Dennoch ist durch den Parser und durch die implementierten Verfahren eine gute Grundlage geschaffen worden, die direkt auf den theoretischen Grundlagen der Logik und Informatik aufbaut. Die Grundlagen der logischen und symbolischen Programmierung sowie Rekursion über Listen lassen sich bereits jetzt mit dem System üben.

Die gewählte interne Repräsentation von Formeln ermöglicht eine einfache Erweiterung und Weiterentwicklung des Systems.
