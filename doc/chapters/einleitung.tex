\chapter{Einleitung}
Schlussfolgerungssysteme entscheiden, ob eine Aussage aus einer Menge von Axiomen folgt. Sie stellen damit eine unmittelbare Anwendung der formalen Logik dar und sind der Kern logischer Programmiersprachen. Als Inferenzmaschinen finden sie Anwendung in der künstlichen Intelligenz.

Aufgrund ihrer Relevanz für praktische Anwendungen sind Schlussfolgerungssysteme in Form der logischen Programmiersprache Prolog Teil der Lehrveranstaltung Angewandte Logik. Den Studenten sollen Grundlagen der logischen und symbolischen Programmierung sowie das Prinzip der Rekursion näher gebracht werden. Nicht logische Bestandteile von Prolog sind nicht Inhalt der Lehrveranstaltung.

Als Alternative für Prolog sollte ein Schlussfolgerungssystem entwickelt werden, das den Fokus auf logische und symbolische Programmierung anstatt auf praktische Relevanz legt. Es sollte einfacher sein, den Zusammenhang zwischen Formeln und Logik-Programmen und so die Relevanz der theoretischen Inhalte der Lehrveranstaltung zu erkennen.

Zu diesem Zweck soll in dieser Arbeit die formale Sprache CL-Reason entwickelt werden, die Nähe zur klassischen Darstellung von Formeln mit dem syntaktischen Komfort von Prolog verbindet. Für diese Sprache wird ein Parser entwickelt, mit dessen Hilfe es möglich ist, Formeln in eine interne Repräsentation zu überführen. Auf dieser internen Repräsentation der Formeln ist nun das automatisierte Durchführen von Schlussfolgerungen gemäß der in der Lehrveranstaltung vorgestellten Techniken möglich.

Die Arbeit behandelt Grundlagen der formalen Logik (Kapitel \ref{logik}) und für Konzeption und Implementierung wichtige Erkenntnisse aus dem logischen Kern der Programmiersprache Prolog (Kapitel \ref{prolog}). Schließlich wird das Vorgehen bei der Konzeption (Kapitel \ref{konzeption}) und der Implementierung (Kapitel \ref{implementierung}) beschrieben. Ein Ausblick beschreibt denkbare und sinnvolle Erweiterungen des entwickelten Systems (Kapitel \ref{ausblick}).
